\documentclass[12pt,a4paper]{article} 
\usepackage{listings}
\usepackage{graphicx}
\usepackage{wrapfig}
\lstloadlanguages{IDL}
\pagestyle{plain}
\parindent=0.0cm

\textheight=25cm
\textwidth=16.3cm

\oddsidemargin=0.0cm
\evensidemargin=0.0cm

\marginparwidth=0.0cm
\marginparsep=0.0cm

\topmargin=0.0cm
\headheight=0.0cm
\headsep=0.0cm

\newcommand{\code}[1]{\texttt{#1}}

\footskip=1.3cm

\begin{document} 

{\center \Huge{Reference documentation for NeuroField}}

\section{Quick guide to using NeuroField}

Using the launcher script NeuroField:
\begin{enumerate}
\item Edit the Makefile. Identify the platform, and comment/uncomment the appropriate COMP and CP directions in Makefile.
\item To execute with only one set of parameters, edit your configuration file in Configs, then run \code{./NeuroField Configs/config file}. For example,
   \code{./NeuroField Configs/conf.emirs}
   The output will be stored in the current directory.

\item To execute with varying parameters, edit your configuration file in Configs, then run './NeuroField Configs/config file object1 to vary ... objectN to vary parameter to vary list of values object2 to vary ... objectM to vary parameter to vary list of values ...' For example,
   ./NeuroField Configs/conf.emirs 'Propagator 1' 'Propagator 2' gamma 10 20 30 40
   ./NeuroField Configs/conf.emirs 'Propagator 1' 'Propagator 2' gamma 10 20 30 40 'Coupling data 1' Nu 1 2 3 4
   In both cases, NeuroField is ran 4 times, with gamma (and Nu) varying in parameters for the specified objects.
   To preserve the sign of the original value, append 's' to the value, eg.
      ./NeuroField Configs/conf.emirs 'Coupling data 6' Nu 1s
      would give Nu=-1 because it is initially -0.00095 in the config file.
   The output will be stored in Output/Parameter=Value directories.

\item To compile NeuroField without running, run 'make'.

\item 'make clean' removes the entire 'Release' folder, so make sure you do not put any user data in there! 

\item Documentation can be found in ./Documentation/ folder. To generate doc.pdf, run 'make Documentation/doc.pdf'. 'make clean' deletes the files created by latex.

\item Tips for vi users: a convenient way to do parameter exploration is to edit the configuration file from ./trunk, then in command mode, do ':!./NeuroField % [additional parameter]'.
\end{enumerate}

\section{Overview of code function}

The code implements the eeg model equations.

\begin{itemize}

\item The code currently runs in one of three modes: (1) Initial run,
(2) restart calculation and (3) default parameters mode.

(1) In inital run mode
code takes input from an neurofield.conf file
initializes the simulation and then takes specified number of integration steps
writing out a neurofield.output file and a neurofield.dump file for restarting.
The input filename can be changed on the command line by using the switch
``-i newinputfilename". Similarly the name of the output filename which defaults
to neurofield.output can be changed via the switch ``-o newoutputfilename". The
dump file which defaults to neurofiled.dump can be changed via ``-d newdumpfilename".

(2) The code dumps sufficient data about parameter values to restart. In
restart mode the simulation will continue where it left off. Restart mode is
accessed by running the code with command ``neurofield restart" rather than
the usual ``neurofield" command on the command line. In order to run the code in restart
mode it is necassary to take the ``neurofield.dump" file which was produced from
the run which will be restarted to be renamed as the
new ``neurofield.conf" file. The new ``neurofield.conf"
file should also be changed to reflect the new number of timesteps etc.
Note:- A restarting ``neurofield.conf" file has different content to an initial run
``neurofield.conf" and can only be used in restart mode.

(3) The default parameters
mode is accessed by running the code with the command ``NeuroField write defaults"
on the command line. The code then writes it own default ``neurofield.conf" file
and uses it for a initial run. The default ``neurofield.conf" file can be 
inspected after the code finishes.

\item One integration step of the model implements the following stages:
1) Firing Response/Stimulus Response. 2) Wave Equation Integration Step
which includes Q delay processing 3) Coupling Response 4) Dendritic Response
5) Afferent Summation.

\item Dendritic Response - is by direct explicit step or Fourth order Runge Kutta.
Dendritic Response solves the following equation:

\begin{equation}
\left\{ \frac{1}{\alpha_{ab}({\bf r},t)\beta_{ab}({\bf r}, t)} \frac{d^2}{dt^2} +
  \left[ \frac{1}{\alpha_{ab}({\bf r}, t)}+ \frac{1}{\beta_{ab}({\bf r}, t)} 
  \right]
  \frac{d}{dt} + 1
\right\}
V_{ab}({\bf r}, t) = P_{ab}({\bf r}, t).
\end{equation}

\item Wave Equation integration is by explicit finite difference (9 point) algorithm
   on a regular square grid. Other parts of code are uneffected by spatial
   geometry so this can be switched to irregular gridding easily. The wave equation
   currently solved is 

\begin{equation}
\left[\frac{1}{\gamma_{ab}^2}\frac{\partial^2}{\partial t^2}+\frac{2}{\gamma_{ab}}\frac{\partial}
{\partial t}+1-r_{ab}^2\nabla^2 \right] \phi_{ab}({\bf r},t)
=Q_b({\bf r},t-\tau_{ab}).
\end{equation}
The patchy propagator has been implemented but has not tested.

\item Two types of coupling response have been implemented.
The first is a relatively simple and most commonly used form which is

\begin{equation}
P_{ab}({\bf r}, t) = \nu_{ab} \phi_{ab}({\bf r}, t).
\label{eq:simplecoupling}
\end{equation}
The second form is a coupling which implements a time evolving coupling $\nu(t)$
according to the Rennie/Clearwater evolution equation. The form is described later
in this document.

\item Parameters $\alpha_{ab}$, $\beta_{ab}$, $\theta_{a}$, $\sigma_{a}$, Q$_{a \: \rm{max}}$,
  $\nu_{ab}$, Effective range $r_{ab}$, $\gamma_{ab}$
  are all set at initialization currently. They have been explicitly
  designed to allow replacement with functional forms allowing evolution
  in time and space as needed.
  
\item Arbitrary numbers of populations are allowed.

\item In order to save memory a connection matrix between the populations is
   specified at run-time. This ensures that where two populations are not
   connected the corresponding wave equation and dendritic response are
   not created in memory.

\item Arbitrary number of stimulus populations are allowed. Stimulus populations
  are populations of neurons which correspond to different stimulus 
  modes. They have no feedback into them from the other populations.
  Three simple stimulus patterns are currently implemented.
  1) Constant in space and time 2) Gaussian noise of constant amplitude
  in space and time 3) Sinusoidal variation in time of stimulus with constant
  amplitude and stimulus is spatially uniform.
  These correspond to the first three stimulus types in Chris Rennie's code.
  Other stimulus patterns are to be written shortly.
  
\end{itemize}
 
\section{Input files}

The input file is a human readable file of the parameters of the simulation.
An input file is generated by the default running procedure of the code. This can be
useful for understanding the layout of the input file. The default file
creates a input file for the standard corticothalmic model. Before
reading the input file description it is worthwhile to 
peruse the default file.

The input file is in a number of sections, first the global simulation information,
next the information for each population is given, then the propagation data,
then the coupling data and finally the output data.

\subsection{Global information}
Each bullet point corresponds to a line of the input file.
\begin{itemize}
\item The first line of the input file is skipped to allow the user to have a title
line
\item {\tt \small \small Nodes per population: 2500} This line contains the number of grid points
in the spatial dimension per population of neurons. The code has been
explicitly designed to have equal number of neurons per population.
\item {\tt \small Number of neural populations: 5} This is the number of neural
populations and includes the number of stimulus modes. It is denoted
$n$ in this document.
\item {\tt \small Number of neural connections: 11} This is the number of neural
populations which are connected together. By definition this number lies in the
range $n$ and $n^2$ where $n$ is the number of neural populations.
\item A blank line is inserted in the file at this point
\item {\tt \small Populations connections matrix 1 2 3 4 ....} The code ignores this
line but it is useful to provide a way of lining up the data in the
connections matrix.
\item Now follows $n$ lines of connection matrix data. On each line the code
looks for colon character followed by a number. There should be $n$ colons
followed by a number on each line. The row number corresponds to the
population ID number of the TO population. The column number corresponds 
to the population ID number of the FROM population of the connection. Zeros
correspond to unconnected populations whilst nonzero number correspond
to connected populations. For easy of use it is suggested that each connection
be numbered in ascending order when reading down columns
as in the default input file. This allows
human crosschecking with later sections of the input file.
\item {\tt \small Number of integration steps: 100000 Skippoints: 49 Deltat: 0.0001}
The parameters are: 
(1) the number of integration steps the code should step forward
during the
run, (2) An optional parameter of how many integration steps should
be skipped before an time step should be outputted, and
(3) the time increment for each time step.

\item {\tt \small Q delay depths: 420 :0 :0 :420 :0} This line is no longer required 
and is ignored.

\item {\tt \small Propagator types 1: Waveeqn 2: Waveeqn 3: Waveeqn 4: Waveeqn 5: Mapping \$
6: Mapping 7: Mapping 8: Waveeqn 9: Waveeqn 10: Mapping 11: Mapping \$} These lines contain
$n$ colons preceeded by an index number and followed by a specification of the propagator
type. The propagation data should appear in the same order as in the connection matrix when the 
connection matrix is read top to bottom and then left to right. Three types of propagator
are available: 1) ``Waveeqn" - is a propagator
implementing the usual wave equation as described in the previous section.
2) ``Mapping" is a propagator which maps $Q_b(t-\tau)$ directly to $\phi_{ab}(t)$. This
propagator is equivalent to cases where spatial spreading is negligible i.e. for a wave equation
where numerically $r_{ab}$ is small and $\gamma_{ab}$ is large. 
3) ``Eqnset" - is the untested Robinson patchy propagator.
4) ``Harmonic" - is a harmonic oscillator propagator i.e. identical with ``Waveeqn" operator 
but with no spatial dependence.
\item {\tt \small Coupling types 1: Modulate 2: Simple 3: Simple 4: Simple 5: Simple 6: Simple \$ 
7: Simple 8: Simple 9: Simple 10: Simple 11: Simple \$} These lines contain
$n$ colons preceeded by an index number and followed by a specification of the synaptic coupling
type. Two types of coupling are available: 1) ``Simple" - is a coupling which does not evolve
with time and is given in Eq. \ref{eq:simplecoupling}. 2) ``Modcouple" - is a Clearwater/Rennie
coupling described in detail later in this document.

\item A blank line finishes this section of the input file
\end{itemize}

\subsection{Population data}
This section contains $n$ repeated population information sections.
Each population is presented in the same order as the connections matrix.
There are two types of neural populations: 1) ordinary populations
and 2) stimulus populations. The two types of populations have different 
input information sections
\subsubsection{Ordinary populations}
Each population information section is as follows. Each bullet point
corresponds to a line of the population information section.
\begin{itemize}
\item The first line is ignored by the code but should be used for
a text description of the population's name eg. 
{\tt \small Population 1 - Excitory neurons}.
\item {\tt \small Initial Q: 8.87145} When the simulation starts an initial
value for the firing rate $Q$ is required.
\item {\tt \small Firing Response Theta: 0.013 Sigma: 0.0038 Qmax: 250.0} These
are the parameters of the population sigmoid firing response. Note the
sigma value accepted by the code is in some Robinson papers it is known as $\tilde{\sigma}$.
It is already scaled  by $\pi / \sqrt{3}$.
\item {\tt \small Number of Dendritic responses: 3} The code uses this line
purely for crosschecking purposes. It is equal to the number of
populations whose connections terminate on the dendrites of the current
population i.e. the number of non-zero elements in the column 
of the connection matrix of this population. We shall denote this number 
$m$
\item Now follow $m$ lines of dendritic response data. Each Dendritic
response data line begins with an optional preamble which is ignored 
by the code. It is strongly suggested that for human readability this
should be of the form {\tt \small Dendritic Response from population 3}. A complete
dendritic response input line is given by

{\tt \small Dendritic Response from population 3 V initial: 0.0106457 alpha: 75.0 beta: 285.0}.

The following three parameters are the initial value of the subvoltage $V_{ab}$
and the parameters $\alpha_{ab}$ and $\beta_{ab}$ of the dendritic response differential
equation. It should be noted the code has a separate dendritic response
differential equation for each different population synapsing on the
dendritic tree. The subvoltage $V_{ab}$ is the voltage in the dendrite due to
incoming connections from population $b$. In some earlier papers all
pulse densities were summed before a single differential equation described
the total voltage in the dendritic tree. This code allows different 
dendritic responses for pulse densities arriving from different populations. If
each dendritic response in a population has the same alpha and beta values
the earlier case is effectively reproduced. If the initial $V_{ab}$ is given by
the initial values of $\nu_{ab}$ and $\phi_{ab}$ via $V_{ab}= \nu_{ab} \phi_{ab}$
as is the case in a steady state condition then for example 
the dendritic response may be given as

{\tt \small Dendritic Response from population 3 V initial: Steady alpha: 75.0 beta: 285.0}.

The ``Steady" short hand feature cannot be used if any of the coupling described
later in this document are of the ModCouple type.

\item A blank line finishes each population information section.
\end{itemize}

\subsubsection{Stimulus populations}
The code identifies stimulus populations as populations which have no 
connections from other populations on their dendritic tree i.e.,
The column for that population contains no nonzero elements.
Each stimulus population information section is as follows. Each bullet point
corresponds to a line of the population information section.
\begin{itemize}
\item The first line is ignored by the code but should be used for
a text description of the population's name eg. 
{\tt \small Population 5 - Stimulus neurons}.
\item {\tt \small Initial Q: 8.87145} When the simulation starts an initial
value for the firing rate $Q$ is required.
\item The final line depends on the type of stimulus the stimulus population
provides. Examples of the current stimulus patterns are provided below.

Pulse stimulus pattern

{\tt \small Stimulus mode: 1 Time to start of stimulus: 0.002 Amplitude: 20
Pulse Duration: 0.02 Pulse repetition period: 0.005}

White noise stimulus pattern (spatially uncorrelated) where Ranseed
is an optional parameter for the random number generator's initial seed defaulting to -98716872.

{\tt \small Stimulus mode: 2 Time to start of stimulus: 0.002 Ranseed:-98716872 Amplitude: 20}

Sinusoidal stimulus pattern -Sinusoidally modulated firing rate

{\tt \small Stimulus mode: 3 Time to start of stimulus: 0.002 Amplitude: 20
Modulation Frequency: 10}

Coherent white noise stimulus pattern (spatially coherent) where Ranseed
is an optional parameter for the random number generator's initial seed defaulting to -98716872.

{\tt \small Stimulus mode: 4 Time to start of stimulus: 0.002 Ranseed:=-98716872 Amplitude: 20}
\end{itemize}

\subsection{Propagation data}
Each bullet point corresponds to a line of the propagation data section.
\begin{itemize}
\item The first line is ignored and is merely a title line {\tt \small Propagation data}.
\item Now follow $n$ lines of propagation data. There is a line of propagation
data for each connection in the connection matrix. The propagation data
should appear in the same order as in the connection matrix when the 
connection matrix is read top to bottom and then left to right. There are three types
of propagators each with their own input parameters
\end{itemize}

\subsubsection{WaveEqn}
This propagator is the wave equation propagator and has input given by

{\tt \small Propagator 2 - Initial Phi: 8.87145 Tauab: 0 Effective Range: 0.08
gamma: 100.0}.

The Initial Phi is the initial value for $\phi_{ab}$ in the
wave equation. Tauab is $\tau_{ab}/ \Delta t$ the delay term in the wave equation
i.e. number of time steps in the delay term. Alternative ways of specifying
the delay term are described below.
Effective range is $r_{ab}$
in the wave equation. The final parameter can be gamma or velocity in the wave equation.
If the initial $phi_{ab}$ is equal to the initial $Q_{ab}$ as occurs in the steady state condition
the propagator input can for example be given as

{\tt \small Propagator 2 - Initial Phi: Steady Tauab: 0 Effective Range: 0.08
velocity: 10.0}.


\subsubsection{Mapping}
{\tt \small Propagator 5  - Tauab: 0 }

This propagator is the mapping propagator where spatial spreading is negligible.
Its form is given by
\begin{equation}
\phi_{ab}({\bf r}, t) = Q_b ({\bf r}, t - \tau_{ab}).
\end{equation}

The input form has only one parameter Tauab (the delay term).
i.e. number of time steps in the delay term. Alternative ways of specifying
the delay term are described below.

\subsubsection{Eqnset}

This is the Robinson patchy propagator. It has not been currently tested.

\subsubsection{Harmonic}
This is a propagator implementing a harmonic oscillator of the form
\begin{equation}
\left[\frac{1}{\gamma_{ab}^2}\frac{d^2}{d t^2}+\frac{2}{\gamma_{ab}}\frac{d}
{d t}+1 \right] \phi_{ab}({\bf r},t)
=Q_b({\bf r},t-\tau_{ab}).
\end{equation}
The input form is given by 

{\tt \small Propagator 2 - Initial Phi: 8.87145 Tauab: 420 gamma: 10.0},

where Tauab is $\tau_{ab}/ \Delta t$ the delay term in the wave equation
i.e. number of time steps in the delay term. Alternative ways of describing
the time delay are specified below. An example of an alternative form is

{\tt \small Propagator 2 - Initial Phi: 8.87145 Tauabt: 0.42 gamma: 10.0},

\subsubsection{Optional Tau forms}

In addition to specifying $\tau$ as a number of discrete steps via the
relation -- Tauab $= \tau_{ab}/ \Delta t$.

Taauabt is the delay term in seconds rather than time steps.
In this case the program automatically rounds to an integer number
of time steps. Care should be taken to understand this implicit rounding.
The simulation code is a finite differencing system and this approximation
is a consequence of this fact.

TauabArray allows specification of the delay term at each grid point
of the simulation allowing a spatially varying delay term. In this form
the delays are specified in time steps. An example of this for a simulation
with only 4 nodes per population is

{\tt \small TauabArray: 420 :430 :425 :410}

TauabtArray is the analog of TauabArray but with delay terms specified in
seconds rather than time steps. Note the same care with regard to
rounding should be taken as for Tauabt.

\subsection{Coupling data}
The coupling data is $n$ lines of coupling data corresponding
to each connection in the connection matrix. The coupling data should appear
in the same order as in the connection matrix when the connection matrix is 
read top to bottom and then left to right. There are two types of coupling
each with their own input form.
A blank line finishes the coupling data section.

\subsubsection{Simple}
For simple synaptic coupling each Coupling data line contains a single parameter Nu e.g.

{\tt \small Coupling data 1 - Nu: 0.0012}.

The parameter Nu ($\nu_{ab}$)
is the synaptic coupling
parameter. It corresponds to the product of the mean synaptic strength $s_{ab}$
and $N_{ab}$ the mean number of connections from cells of type $b$ to
cells of type $a$.

\subsubsection{Modcouple}

The modulated synaptic coupling implements the Clearwater/Rennie modulation
equation. The modulation is given by:

\begin{equation}
\nu (t) = \nu_0 \left[ ( 1 - \nu_{scal} ) e^{h(t)/k} + \nu_{scal} \right]
\end{equation}
where $h(t)$ is the time low passed filtered form of the neuromodulators concentration
$C(t)$ as given by
\begin{equation}
\left\{ \frac{1}{\mu \lambda} \frac{d^2}{dt^2} +
  \left[ \frac{1}{\nu}+ \frac{1}{\lambda} 
  \right]
  \frac{d}{dt} + 1
\right\}
h(t) = C(t).
\end{equation}
The neuromodulator's concentration is in turn given by a user chosen
stimulus form analogous to stimulus populations.

The remainder of the input form is specification of the output for $\nu$.
This takes an analogous form to the usual output data lines. The $\nu$
data is output to a file with filename ``neurofield.synaptout.xx" where xx is an index number
of the coupling.
\\
\\
An example Modcouple input form is given by
\\
\\
{\tt \small Coupling data 1  - Nuzero: 0.0002 Nuscal: 0.02 Mu: .1 Lambda: 1 k: .000001 \\
 Concentration mode: 1  Time to start of Concentration: .01 Amplitude: .0001\\
 Pulse Duration: .2 Pulse repetition period: 40 \\
 Number of traces: 100  Single/All: All nodes}


\subsection{Output data}
Currently the code output a time series of any particular node in any population.
The first line of this section contains an optional preamble and a parameter
of how many times series to be outputted e.g.

{\tt \small Output data - Number of traces : 4}.

Following the first line
each trace is specified on a separate line in the form:

{\tt \small Wave Equation Number :1 Single/All Single node number :1}.

A short cut to output all the traces in a population is given the the
input form

{\tt \small Wave Equation Number : 11 Single/All nodes: All nodes}


\section{Known bugs or flaws} 

\begin{itemize}
    
\item Code only runs for initial time set to zero. This only effects the
stimulus routines.

\item Inadequate range of stimulus patterns currently implemented.

\item Documentation needs to be worked on.

\item Code needs more commenting.

\end{itemize}

\end{document}
